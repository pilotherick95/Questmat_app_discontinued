\documentclass[12pt]{article}
\usepackage[utf8]{inputenc}
\usepackage[T1]{fontenc}
\usepackage[portuguese]{babel}
\usepackage{amsmath}
\usepackage{amsfonts}
\usepackage{amssymb}
\usepackage{graphicx}
\usepackage{multicol}
\usepackage{multirow}
\usepackage{array}
\usepackage{booktabs}
\usepackage{xcolor}
\usepackage{times}
%\usepackage[none]{hyphenat}
	
	\graphicspath{{./images/}}
	

\begin{document}

\author{Lucas Herick P.S.}
\title{QUESTMAT:TRIVIA}
%\logo{}
\date{II ENC PIRP \ I DOC PIBID/PIRP}
%\subject{}
%\setbeamercovered{transparent}
%\setbeamertemplate{navigation symbols}{}	
	
\thispagestyle{empty}
\begin{center}
	{\large \textbf{ QUESTMAT: UMA PROPOSTA DE GAMIFICAÇÃO DA MATEMÁTICA BÁSICA}
}
\end{center}

\begin{flushright}
\textbf{	Lucas Herick Pereira da Silva\\
Sandra William Marques\\
Fabiano dos Santos Souza }
\end{flushright}


\noindent
O presente instrumento - QuestMat Trivia - foi desenvolvido com o auxílio de assets em linguagem computacional C\# com o Ambiente de Desenvolvimento Integrado Unity, que é um ambiente internacionalmente conhecido como um dos melhores softwares de desenvolvimento de jogos 2D e 3D para sistemas operacionais Windows, Linux, Android, iOS, entre outros. O objetivo inicial é que alunos façam uso da aplicação para trabalho e treinamento nas categorias que sentirem maior dificuldade no meio escolar, e introduzir também um ambiente colaborativo entre docentes e discentes. Conforme cita (Brasil, 1998), a gamificação de um conteúdo muito auxilia no Ensino-aprendizagem da matemática. (Grando, 2001) também cita que a gamificação também pode se tornar um eixo motivador na memorização dos conteúdos. Tecnicamente, com o advento da tecnologia, também era necessário desenvolver algo que registrasse pleno funcionamento em algo que uma parte majoritária dos alunos possuíssem, no caso, seus aparelhos móveis, causando assim, uma motivação extra. Conforme pesquisa realizada pela \textit{StatCounter}, os sistemas operacionais mais utilizados no mundo são o Android, sistema operacional do Google, e o Windows, sistema operacional da Microsoft. 
Por isso, visando atingir o maior público, foram desenvolvidas versões para ambos os sistemas.  A metodologia de criação foi analisar o comportamento dos alunos durante período de observação, tomando nota das maiores dificuldades apresentadas pelos mesmos, tornando este o processo de seleção natural das categorias para serem imputadas na aplicação, sem que a interface deixe de ser amigável ao usuário.     

%--------------------------------------------------------------
\end{document}